%For decades~\cite{PhysRev.81.165}, it has been known that the nucleon-nucleon potential has a short-range repulsive core, which is responsible for the stability of strongly interacting matter. However, a description of the repulsive core remains largely unconstrained and our understanding of QCD dynamics at short distances ($\leq 0.5\mathrm{~fm}$) largely incomplete~\cite{Sargsian:2014bwa}. 

The deuteron is the simplest composite nuclear system, and in many ways it is as important to understanding bound states in QCD as the hydrogen atom was to understanding bound systems in QED.  Our experimental and theoretical understanding of the deuteron remains unsatisfying. 

Due to their small size and simple structure, tensor polarized deuterons are ideal for studying nucleon-nucleon interactions. Tensor polarization enhances the D-state contribution, which compresses the deuteron~\cite{Forest:1996kp}, 
%in a toroid as shown in Fig.~\ref{fig:dpol-shape}, 
making the system more sensitive to short-range QCD effects. Understanding the nucleon-nucleon potential of the deuteron is essential for understanding short-range correlations as they are largely dependent on the tensor force~\cite{Arrington:2011xs}. We can resolve the short-range structure of nuclei on the level of nucleon and hadronic constituents by utilizing processes that transfer to the nucleon constituents both energy and momentum larger than the scale of the NN short-range correlations, particularly at $Q^2>1~(\mathrm{GeV}/c)^2$.


By taking a ratio of cross sections from electron scattering from tensor-polarized and unpolarized deuterons, the S and D-wave states can be disentangled, leading to a fuller understanding of the repulsive nucleon core. A measurement of $A_{zz}$ is sensitive to the $\frac{D^2-SD}{S^2+D^2}$ ratio and it's evolution with increasing minimal momentum of the struck nucleon. Originally calculated by L. Frankfurt and M. Strikman~\cite{Frankfurt:1988nt}, this has recently been revisited by M. Sargsian, who calculated $A_{zz}$ in this region using a light cone approach and a virtual nucleon approach. The calculations vary by up to a factor of 2, and can be experimentally determined at the $3-6\sigma$ level as discussed in this letter.

%A measurement of $A_{zz}$ will put an experimental constraint on the D-state admixture in models of the deuteron wavefunction.


For the lower $Q^2$ region, W. Van Orden has calculations in progress using different nucleon-nucleon potentials, as well as different prescriptions for handling the reactions mechanisms~\cite{vanorden-convo}. Similar calculations are currently being finalized for the approved
D($e,e'p)n$ at high $Q^2$, high $p_m$ experiment. Once completed, he will turn his attention to tensor polarization observables in the low $Q^2$ region.
% By measuring $A_{zz}$ over a range of $Q^2$, we can  access the evolution of the tensor state dominance, and thus the expected evolution of short range correlation effects.

Additionally, measuring $A_{zz}$ in the quasi-elastic region will fill a gap in measurements performed on tensor polarized deuterium. It is directly proportional to the observable used in the elastic region to measure $T_{20}$, 
by $A_{zz} \propto T_{20}$. Due to the large acceptance of the SHMS spectrometer, we will be taking data in the $x = 2$ elastic $T_{20}$ range as well that, at low $Q^2$, may be able to resolve the discrepancy of $T_{20}$ that was measured at JLab in Halls A and C~\cite{Abbott:2000fg}. This will be investigated further when the full proposal is submitted.  
In the deep inelastic region, $A_{zz}$ will soon be measured to extract the tensor structure function $b_1$ by the relation $A_{zz} \propto \frac{b_1}{F_1^D}$. Not only will measuring $A_{zz}$ in the quasi-elastic region provide information necessary for understanding the properties of the deuteron and contribution from the tensor force, but it will be the first experiment to bridge a gap in measurements of electron scattering from tensor-polarized deuterons.



%Early calculations indicate a discrepancy in $A_{zz}$ from using different NN potentials in the light cone model, but need to be investigated further. Additionally, calculations at low momentum transfer indicate that similar discrepancies will show up in the non-relativistic, low momentum transfer region. These theoretical developments will be investigated as further motivation for the upcoming proposal.
%Jefferson Lab is the ideal place to investigate tensor structure in a deuteron target at intermediate and large $x$.  We describe such a measurement in this proposal.

%\subsection{Deuteron Wavefunction}

%\begin{figure}
%\centering
%\includegraphics[width=0.5\textwidth]{figs/deuteron_states.eps}
%\caption{\label{fig:dpol-shape}
%Equidensity lines of the deuteron in its two spin projections, $M_J=\pm 1$ and $M_J=0$, respectively. Reproduced from~\cite{Carlson:1997qn,Forest:1996kp}.
%}
%\end{figure}

