\subsection{Interest from Theorists}

The potential measurement described in this letter has stirred interest in a number of theorists who are working on calculations. Many of these are on-going and weren't completed in time for the PAC 42 deadline, but are expected to be completed in the coming months to further the physics motivation of a measurement of $A_{zz}$ in the quasi-elastic region at various momentum-transfer.

Discussed in the previous two sections, light cone and virtual nucleon calculations of $A_{zz}$ are the furthest along, with current calculations presented within the kinematics of interest. This work was done by M.~Sargsian~\cite{misak-convo} and M.~Strikman~\cite{strikman-convo}. Early calculations of the M.~Strikman light-cone model indicate a potential measurable discrepancy based on the input of different NN potentials, but the results are very preliminary and require further investigation. S.~Liuti has agreed to join in this theoretical effort, stating ``This is an important measurement to know, and should be calculated more thoroughly."~\cite{liuti-convo}

Models involving 6-quark calculations of quasi-elastic $A_{zz}$ can be calculated by G.~Miller~\cite{miller-convo} and have been motivated by the collaboration. In his own words, he states ``This measurement was a highlighted need early at JLab. A new measurement at higher $Q^2$ would be very interesting. In principle such could test my model. I could calculate the influence of my 6-quark configurations on elastic scattering."

Continuing his interest from DIS $b_1$ calculations, W.~Cosyn is developing calculations of the quasi-elastic contribution to inclusive deuteron scattering, which will be the dominant contribution in the $x>1$ regime~\cite{cosyn-convo}. His calculations can be modified to include $A_{zz}$ and expects to have them within the coming months. In his words, ``I hope to do some 
calculations soon and could easily do them for the kinematics in your 
proposal."


In addition, W. Van Orden has calculations in progress using different nucleon-nucleon potentials, as well as different prescriptions for handling the reactions mechanisms~\cite{vanorden-convo}.  Similar calculations are currently being finalized for the approved
D($e,e'p)n$ at high $Q^2$, high $p_m$ experiment. Once completed, he will turn his attention to tensor polarization observables in the low $Q^2$ region and investigate the effects from differing NN potentials.