

In the quasi-elastic region, the tensor-polarized target asymmetry, $A_{zz}$, which is used to extract $b_1$ in the DIS region and is proportional to $T_{20}$ in the elastic region through the $\stackrel{\leftrightarrow}{\mathrm{D}}$($e,e'$)X channel, can be used to extract information on the ratio of the S- and D-states in the deuteron wave function. This ratio is currently not well constrained experimentally and is an important quantity to determine for understanding tensor effects, such as NN short range correlations, that is most clearly manifested in the scattering off the polarized deuteron due to a strong dependence of the S/D ratio on the nucleon momentum.

In the quasi-elastic region, $A_{zz}$ was first calculated in 1988 by Frankfurt and Strikman, using the Hamada-Johnstone and Reid soft-core wave functions \cite{Frankfurt:1988nt}. Recent calculations by {M.~Sargsian} revisit $A_{zz}$ in the $x>1$ range using virtual-nucleon and light-cone methods, which differ by up to a factor of two \cite{MISAK}. 

An experimental determination of $A_{zz}$ could be performed utilizing identical equipment identical as the E13-12-011 $b_1$ experiment at three different $Q^2$ values over the course of \productiondays days, with \overheaddays additional days of overhead. The measurements are less sensitive to systematic uncertainties than E13-12-011, such that this experiment could additional be used to understand the in-beam conditions of a tensor polarized target.