We propose the first measurement of the tensor asymmetry $A_{zz}$ in the quasi-elastic region through the tensor polarized D($e,e'$)X channel; an asymmetry that is sensitive to the nucleon-nucleon potential.  Previous measurements of $A_{zz}$ have been used to extract $b_1$ in the DIS region and $T_{20}$ in the elastic region. In the quasi-elastic region, $A_{zz}$ data will be used to compare light cone calculations with variation nucleon-nucleon calculations, and is an important quantity to determine for understanding tensor effects, such as the dominance of $pn$ correlations in nuclei.


In the quasi-elastic region, $A_{zz}$ was first calculated in 1988 by Frankfurt and Strikman, using the Hamada-Johnstone and Reid soft-core wave functions~\cite{Frankfurt:1988nt}. Recent calculations by
M. Sargsian revisit $A_{zz}$ in the $x > 1$ range using virtual-nucleon and light-cone methods, which differ by up to a factor of two~\cite{MISAK} and can be discriminated experimentally at the $3-6\sigma$ level. This potential measurement has stirred the interest of a number of theorists, and will be proposed in full as calculations solidify.

An experimental determination of $A_{zz}$ can be performed utilizing the same equipment as the E13-12-011 $b_1$ experiment.  Three different $Q^2$ values can be measured over the course of \productiondays days, with \overheaddays additional days of overhead. The measurements are less sensitive to systematic uncertainties than E13-12-011, so this experiment could in parallel be utilized to better understand the in-beam conditions and time-dependent systematic effects of a tensor polarized target for the $b_1$ experiment.




%We propose the first measurement of the tensor asymmetry $A_{zz}$ in the quasi-elastic region through the $\stackrel{\leftrightarrow}{\mathrm{D}}$($e,e'$)X channel to determine information on the tensor portion of the deuteron wavefunction. Previous measurements of $A_{zz}$ have been used to extract $b_1$ in the DIS region and $T_{20}$ in the elastic region. In the quasi-elastic region,  $A_{zz}$ can be used to extract the ratio of the S and D-states in the deuteron wave function. This ratio is currently not well constrained experimentally and is an important quantity to determine for understanding tensor effects, such as NN short range correlations, and is most clearly manifested in the scattering off the polarized deuteron due to a strong dependence of the S/D ratio on the nucleon momentum.

%In the quasi-elastic region, $A_{zz}$ was first calculated in 1988 by Frankfurt and Strikman, using the Hamada-Johnstone and Reid soft-core wave functions \cite{Frankfurt:1988nt}. Recent calculations by {M.~Sargsian} revisit $A_{zz}$ in the $x>1$ range using virtual-nucleon and light-cone methods, which differ by up to a factor of two \cite{MISAK} and can be discriminated experimentally at the $3-6 \sigma$ level.

%An experimental determination of $A_{zz}$ could be performed utilizing identical equipment identical as the E13-12-011 $b_1$ experiment at three different $Q^2$ values over the course of \productiondays days, with \overheaddays additional days of overhead. The measurements are less sensitive to systematic uncertainties than E13-12-011, such that this experiment could additionally be utilized to understand the in-beam conditions and time-dependent systematic effects of a tensor polarized target.
