\section{Summary}

%The ratio of the S- and D-wave state of the deuteron is not well constrained by current wavefunction models, which is particularly important in describing tensor-related effects such as short range correlations. 
We have investigated the possibility of making high precision measurements of the quasi-elastic tensor asymmetry $A_{zz}$.  By covering the kinematic range from the QE peak ($x=1$) up to elastic scattering ($x=2$), we expect that this data will provide valuable new insights about the high momentum components of the deuteron wavefunction. We are actively working with several theorists to get state-of-the-art calculations of light cone, virtual nucleon, and six-quark models. Additional calculations are being performed that include final-state interactions, and low $Q^2$ sensitivity to NN potentials.  It is important to note that this is the same kinematic region that has been shown to be correlated with the EMC effect via the $x>1$ A/D ($e,e'$) results. 

We have found that with \productiondays days of beam and an additional \overheaddays days of overhead, $A_{zz}$ can be measured with high precison at $Q^2=1.5$, $0.7$, and $0.3~(\mathrm{GeV}/c)^2$ in Hall C using identical equipment as the upcoming $b_1$ measurement while being less sensitive to systematic uncertainties. In addition, it will fill a gap in measurements of $A_{zz}$ between the $T_{20}\propto A_{zz}$ elastic measurements and the $b_1\propto \frac{A_{zz}}{F_1^d}$ deep-inelastic measurements.  
