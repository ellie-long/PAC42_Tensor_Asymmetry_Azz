


The deuteron is the simplest nuclear system, and in many ways it is as important to understanding bound states in QCD as the hydrogen atom was to understanding bound systems in QED.  Unlike it's atomic analogue, our understanding of the deuteron remains unsatisfying both experimentally and theoretically.  

Through electron scattering on tensor-polarized deuterons, the S- and D-wave states can be disentangled, leading to a fuller understanding of the repulsive nucleon core. 

Understanding the nucleon-nucleon potential of the deuteron is essential for understanding short-range correlations. To resolve the short-range structure of nuclei on the level of nucleon and hadronic constituents, we need processes that transfer to the nucleon constituents both energy and momentum larger than the scale of the NN short range correlations. By scanning over a large range of $Q^2$, we can measure how these processes begin to dominate the tensor asymmetry $A_{zz}$.



%Jefferson Lab is the ideal place to investigate tensor structure in a deuteron target at intermediate and large $x$.  We describe such a measurement in this proposal.
