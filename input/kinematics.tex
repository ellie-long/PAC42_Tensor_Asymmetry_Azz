\subsection{Kinematics}
\begin{table}
\begin{center}
%\begin{tabular}{c|c|c|c|c|c|c|c}
%& $E_0$ & $Q^2$    &  $W$  	& $E'$  &    $\theta_{e'}$  &  Rates   & PAC Time   \\
%& (GeV) & (GeV$^2$)  			& (GeV) 			& (GeV)  &     (deg.)   &   (kHz)  & (hours) \\
%\multicolumn{2}{|c|}{$\times 10^{-2}$}
%\hline\hline
%Spec    E0		Q2		W		P0		 Theta		 Rates		Time
%SHMS & 8.8	&  1.5	&  0.46	&  8.36	&    8.2 	&    0.55	&   600 \\
%SHMS & 6.6	&  0.7	&  0.60	&  6.50	&    8.2 	&    4.08	&   90 \\
%SHMS & 2.2	&  0.3	&  0.87	&  2.11	&    14.4 	&    3.73	&   30 \\
%HMS  & 2.2	&  0.3	&  0.86	&  2.11	&    14.9	&    4.65	&   30 \\  

\begin{tabular}{c|c|c|c|c|c|c}
& $E_0$ & $Q^2$    	& $E'$  &    $\theta_{e'}$  &  Rates   & PAC Time   \\
& (GeV) & (GeV$^2$)  & (GeV)  &     (deg.)   &   (kHz)  & (hours) \\
%\multicolumn{2}{|c|}{$\times 10^{-2}$}
\hline\hline
%Spec    E0		Q2		P0		 Theta		 Rates		Time
SHMS & 8.8	&  1.5	&  8.36	&    8.2 	&    0.43	&   600 \\
SHMS & 6.6	&  0.7	&  6.50	&    8.2 	&    3.19	&   90 \\
SHMS & 2.2	&  0.3	&  2.11	&    14.4 	&    3.73	&   30 \\
HMS  & 2.2	&  0.3	&  2.11	&    14.9	&    2.92	&   30 \\  

\hline\hline
\end{tabular}
\caption{\label{RATES1}Summary of the central kinematics and physics rates using the Hall C  spectrometers.}
\end{center}
\end{table}


\begin{table}
\begin{center}
\begin{tabular}{c|ccc|ccc|ccc}
 ~ & \multicolumn{3}{|c}{$Q^2=1.5\mathrm{~(GeV/}c)^2$} & \multicolumn{3}{|c}{$Q^2=0.7\mathrm{~(GeV/}c)^2$} & \multicolumn{3}{|c}{$Q^2=0.3\mathrm{~(GeV/}c)^2$} \\
 \hline
  $x$  & $f_{dil}$ & $\delta A_{zz}^{stat}$ & $\delta A_{zz}^{sys}$ & $f_{dil}$ & $\delta A_{zz}^{stat}$ & $\delta A_{zz}^{sys}$ & $f_{dil}$ & $\delta A_{zz}^{stat}$ & $\delta A_{zz}^{sys}$ \\
  &     & $\times 10^{-2}$  & $\times 10^{-2}$  &    & $\times 10^{-2}$  & $\times 10^{-2}$ &    & $\times 10^{-2}$  & $\times 10^{-2}$ \\
\hline\hline
%       |         Q2=1.5         |      Q2=0.7          |      Q2=0.3
%  x  	   fdil 	   dAzz	 dAzzSys  fdil 	 dAzz   dAzzSys  fdil   dAzz	 dAzzSys
 0.80	&  0.205	 & 0.58	& 2.03	& 0.175	 & 0.71	& 0.74 & 0.298 & 0.46 & 0.74 \\
 0.90	&  0.274	 & 0.44	& 0.58 	& 0.375	 & 0.31	& 1.68 & 0.462 & 0.29 & 1.68 \\
 1.00	&  0.507	 & 0.23	& 1.01 	& 0.518	 & 0.22	& 0.02 & 0.521 & 0.27 & 0.02 \\
 1.10	&  0.333	 & 0.47	& 0.21 	& 0.409	 & 0.35	& 1.99 & 0.431 & 0.40 & 1.63 \\
 1.20	&  0.174	 & 1.28	& 2.36 	& 0.264	 & 0.74	& 3.98 & 0.301 & 0.74 & 3.25 \\
 1.30	&  0.120	 & 2.63	& 6.28 	& 0.174	 & 1.42	& 5.98 & 0.193 & 1.34 & 4.88 \\
 1.40	&  0.108	 & 4.08	& 10.2 	& 0.156	 & 2.46	& 7.97 & 0.144 & 2.16 & 6.50 \\
 1.50	&  0.096	 & 6.56	& 12.7	& 0.133	 & 3.31	& 9.96 & 0.100 & 3.61 & 8.13 \\
 1.60	&  0.096	 & 8.96	& 12.8 	& 0.110	 & 5.49	& 12.0 & 0.086 & 4.62 & 9.75 \\
 1.70	&  0.095	 & 12.1	& 10.7 	& 0.096	 & 7.88	& 13.9 & 0.063 & 7.13 & 11.4 \\
 1.80	&  0.096	 & 15.5	& 7.18 	& 0.096	 & 10.3	& 14.0 & 0.056 & 8.87 & 13.0 \\
\hline\hline
\end{tabular}
\caption{\label{RATES2}Summary of the expected statistical uncertainty after combining overlapping x-bins.  Values represent the statistics weighted average of all events that satisfy our kinematic cuts. }
\end{center}
\end{table}


%\begin{table}
%\begin{center}
%\begin{tabular}{c|c|c|c|c}
% $Q^2$ & $x$  & $f_{dil}$ & $\delta A_{zz}^{stat}$ & $\delta A_{zz}^{sys}$ \\
%(GeV$^2$)  &     &    & $\times 10^{-2}$  & $\times 10^{-2}$ \\
%\hline\hline
%%	Q2		  x  	   fdil 	   dAzz	  dAzzSys
%		&	 0.80	&  0.205	 & 0.52	& 0.53 \\
%		&	 0.90	&  0.274	 & 0.39	& 1.20 \\
%		&	 1.00	&  0.507	 & 0.21	& 0.05 \\
%		&	 1.10	&  0.333	 & 0.42	& 1.75 \\  
%		&	 1.20	&  0.174	 & 1.13	& 3.51 \\  
%	1.5	&	 1.30	&  0.120	 & 2.32	& 5.26 \\  
%		&	 1.40	&  0.127	 & 3.61	& 7.01 \\  
%		&	 1.50	&  0.096	 & 5.81	& 8.77 \\              
%		&	 1.60	&  0.096	 & 7.93	& 10.0 \\  
%		&	 1.70	&  0.095	 & 10.7	& 10.0 \\  
%		&	 1.80	&  0.096	 & 13.7	& 10.0 \\  	    
%\hline
%		&	 0.80	&  0.175	 & 0.63	& 0.53 \\
%		&	 0.90	&  0.375	 & 0.27	& 1.20 \\
%		&	 1.00	&  0.518	 & 0.19	& 0.05 \\
%		&	 1.10	&  0.409	 & 0.31	& 1.42 \\  
%		&	 1.20	&  0.264	 & 0.66	& 2.85 \\  
%	0.7 &	 1.30	&  0.174	 & 1.26	& 4.27 \\  
%		&	 1.40	&  0.156	 & 2.18	& 5.69 \\  
%		&	 1.50	&  0.170	 & 2.93	& 7.12 \\              
%		&	 1.60	&  0.110	 & 4.86	& 8.54 \\  
%		&	 1.70	&  0.096	 & 6.97	& 9.96 \\  
%		&	 1.80	&  0.096	 & 9.13	& 10.0 \\  	    
%\hline
%		&	 0.80	&  0.298	 & 0.41	& 0.53 \\
%		&	 0.90	&  0.462 & 0.25	& 1.20 \\
%		&	 1.00	&  0.521 & 0.24	& 0.01 \\
%		&	 1.10	&  0.431 & 0.35	& 1.16 \\  
%		&	 1.20	&  0.301 & 0.65	& 2.32 \\  
%	0.3 &	 1.30	&  0.193 & 1.18	& 3.48 \\  
%		&	 1.40	&  0.144 & 1.91	& 4.64 \\  
%		&	 1.50	&  0.100 & 3.19	& 5.81 \\              
%		&	 1.60	&  0.086 & 4.09	& 6.97 \\  
%		&	 1.70	&  0.063 & 6.30	& 8.31 \\  
%		&	 1.80	&  0.056 & 7.72	& 9.29 \\  	    
%\hline\hline
%\end{tabular}
%\caption{\label{RATES2}Summary of the expected statistical uncertainty after %combining overlapping x-bins.  Values represent the statistics weighted average %of all events that satisfy our kinematic cuts. }
%\end{center}
%\end{table}





\label{EXP}
We propose to measure the tensor asymmetry $A_{zz}$ for $\XMIN<x<\XMAX$, $\QMIN$~(GeV/$c)^2 < Q^2 <\QMAX$~(GeV/$c)^2$, and $\WMIN < W < \WMAX$~GeV after applying kinematics cuts. Fig.~\ref{kincov} shows the planned kinematic coverage utilizing the Hall C HMS and SHMS spectrometers at forward angle.
%where production of inelastic final states is expected to be negligible. 

\begin{figure}
\begin{center}
\includegraphics[width=\textwidth]{figs/Pzz_30_all_q2_w.eps}
%\includegraphics[width=0.49\textwidth]{figs/Pzz_30_all_q2.eps}~~
%\includegraphics[width=0.49\textwidth]{figs/Pzz_30_all_w.eps} %\includegraphics[width=0.49\textwidth]{figs/kine/Pzz_30_eprime.eps}
%\includegraphics[width=0.49\textwidth]{figs/kine/Pzz_30_theta_eprime.eps}~~ 

\caption{\label{kincov} Kinematic coverage for central spectrometer settings at $Q^2=1.5~(\mathrm{GeV}/c)^2$ (A), $0.7~(\mathrm{GeV}/c)^2$ (B), and $0.3~(\mathrm{GeV}/c)^2$ (C).  The HMS is only used for setting C, and its coverage largely falls under the SHMS coverage. The grey regions are not included in our statistics estimates since they fall outside of $\XMIN < x < \XMAX$. Darker shading represents areas with higher statistics, and the dotted line in the $W$ plot indicates nucleon mass. }
\end{center}
\end{figure}


The polarized \TARGET target is discussed in section~\ref{POLTARGSEC}.  The magnetic field of the target will be held constant along the beamline at all times, while the target state is alternated between a polarized and unpolarized state.
The tensor polarization and packing fraction used in the rates estimate are \PZZ\% and \PF, respectively. 
The dilution fraction in the range of this measurement is shown in Fig.~\ref{fdil_plot}.
With an incident electron beam current of \CURRENT nA, the expected deuteron luminosity is \LUMI.
%$1.57\times 10^{35}$~cm$^{-2}}$s$^{-1}$.
%$?.??\times 10^{35}$~cm$^{-2}$s$^{-1}$.

The momentum bite and the acceptance were assumed to be $\Delta P = \pm 8\%$ and $\Delta\Omega = 5.6$~msr for the HMS, and $\Delta P= ^{+20\%}_{-8\%}$ 
%$-8<\Delta P <+20\%$
and $\Delta\Omega =4.4$~msr for the SHMS. 
%
For the choice of the kinematics,
special attention was taken onto the angular and momentum limits of the spectrometers: for the
HMS, $10.5^{\circ} \le \theta \le 85^{\circ}$ and $1 \le P_0 \le 7.3$ GeV/c, and for the SHMS,
$5.5^{\circ} \le \theta \le 40^{\circ}$ and $2 \le P_0 \le 11$ GeV/c. In addition, the
opening angle between the spectrometers is physically constrained to be larger than 17.5$^{\circ}$.
The dilution factors and projected uncertainties in $A_{zz}$ are summarized in Table~\ref{RATES2} and displayed in Fig.~\ref{PROJ}.  

\begin{figure}
\begin{center}
\includegraphics[width=0.65\textwidth]{figs/Pzz_30_fdil_all.eps} 
\caption{\label{fdil_plot}Projected dilution factor covering the entire $x$ range to be measured using a combination of P. Bosted's~\cite{Bosted:2012qc} and M. Sargsian's~\cite{misak-convo} code for the SHMS and HMS.}
\end{center}
\end{figure}

\begin{figure}
\begin{center}
%\includegraphics[width=0.45\textwidth]{figs/plots0705/b1_proj_newmiller_lin.eps}
%\hspace{0.5cm}
\includegraphics[width=0.75\textwidth]{figs/Pzz_30_q2_15_Azz_w_misak.eps} \\
\includegraphics[width=0.75\textwidth]{figs/Pzz_30_q2_03_07_Azz_fs.eps} 
\caption{\label{PROJ}Projected statistical errors for the tensor asymmetry $A_{zz}$ with \productiondays days of beam time. The band represents the systematic uncertainty. Also shown for $Q^2=1.5~(\mathrm{GeV}/c)^2$ are calculations provided by M. Sargsian for using a light cone and virtual nucleon model, and for $Q^2=0.3$ and $0.7~(\mathrm{GeV}/c)^2$ a modified Frankfurt and Strikman model~\cite{Frankfurt:1988nt} that estimates the peak shifts in $x$ expected due to the SRC scaling changing with $Q^2$~\cite{Frankfurt:2008zv}.
}
\end{center}
\end{figure}

A total of \productiondays days of beam time is requested for production data, with an additional \overheaddays days of expected overhead.



\clearpage

%





