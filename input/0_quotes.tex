``The most direct evidence for tensor correlations in nuclei comes from measurements of the deuteron structure functions and tensor polarization by elastic electron scattering~\cite{Gilman:2001yh}. In essence, these measurements have mapped out the Fourier transforms of the charge densitites of the deuteron in states with spin projections $\pm1$ and 0, showing that they are very different."~-R. Schiavilla, et al.~\cite{Schiavilla:2006xx}


``The cross section for the double scattering process can be written as~\cite{Arnold:1979cg}
\begin{dmath}
	\frac{d\sigma}{d\Omega d\Omega_2} = \left. \frac{d\sigma}{d\Omega d\Omega_2}\right|_0 \left[1 + \frac{3}{2}hp_xA_y\sin{\phi_2} + \frac{1}{\sqrt{2}}t_{20}A_{zz} - \frac{2}{\sqrt{3}}t_{21}A_{xz}\cos{\phi_2}+\frac{1}{\sqrt{3}}t_{22}\left( A_{xx} - A_{yy} \right) \cos{2\phi_2}  \right]
\end{dmath}
where $h=\pm 1/2$ is the polarization of the incoming electron beam, $\phi_2$ the angle between the two sattering planes (defined in the same way as the $\phi$ shown in figure 24) and $A_y$ and the $A_{ij}$ are the vector and tensor analysing powers of the second scattering. Although there is a $p_z$ component to the vector polarization, the term is omitted from equation (25) as there is no longitudinal vector analysing power; without spin precession, this term cannot be determined."~-R. Gilman and F. Gross~\cite{Gilman:2001yh}

``Accurate [form factor] measurements require that $Q^2$ be known accurately since $A$ and $B$ vary rapidly with $Q^2$. Energy or angle offsets of a few times $10^{-3}$ could lead to $Q^2$ being off by up to $0.5\%$. For both $A$ and $B$, this leads to offsets that increase with $Q^2$, reaching about 2\% at $Q^2=1\mathrm{~GeV}^2$ and 4\% at $Q^2=6\mathrm{~GeV}^2$."~-R. Gilman and F. Gross~\cite{Gilman:2001yh}

``The body of [$A$] data, aside from the lowest $Q$ Orsay point, suggests the correctness of the Saclay measurements. Theoretical predictions span the range between the two data sets, and do not help to determine which is correct. Thus, a new high-precision experiment in this [higher] $Q^2$ range appears desirable."~-R. Gilman and F. Gross~\cite{Gilman:2001yh}

``We are forced to conclude that these high $Q^2$ [form factor] measurements \emph{cannot be explained by nonrelativistic physics and present very strong evidence for the presence of interaction currents, relativistic effects or possibly new physics}."~-R. Gilman and F. Gross~\cite{Gilman:2001yh}

``It is now known that the tensor part of the one-pion exchange interaction is too strong to be treated pertubatively, and recent work has focused on how to include the singular parts of one-pion exchange in the most effective manner~\cite{Phillips:1999hh,Phillips:1999am,Walzl:2001vb}"~-R. Gilman and F. Gross~\cite{Gilman:2001yh}

``But a principal motivation for using the front-form is that it is a natural choice at very high momentum, where the interactions single out a preferred direction (the beam direction) and the dynamics evolves along the light-front in that direction. The disadvantage is that the generators that contain dynamical quantities ar $H_{-}$ and $J^i$, and this means that angular momentum conservation must be treated as a dynamical constraint."~-R. Gilman and F. Gross~\cite{Gilman:2001yh}

``Calculations based on quark degrees of freedom must confront the fact that the deuteron is at least a six-quark system. Since the six quarks are identical (because of internal symmetries) the system must be antisymmetrized, and it is not clear that the nucleon should retain its identity in the presence of another nucleon."~-R. Gilman and F. Gross~\cite{Gilman:2001yh}

``It is clear that much more work will be needed to clarify the various physics issues, before a convergent scheme is established for treating the e.m. and strong interaction physics properly."~-J.A. Tjon~\cite{Tjon.48218}

``... there is no clearly correct way to isolate the structure of the nucleon from the structure of the bound state. In model calculations these issues can be handled by separating the problem into two regions: at large separations ($R>R_c$) it is assumed that the system separates into two nucleons interacting through one pion exchange, and at small distances ($R<R_c$) the system is assumed to coalesce into a six-quark bag with all the quarks treated on an equal footing."~-R. Gilman and F. Gross~\cite{Gilman:2001yh}

``It turns out that this leading twist pQCD estimate is $10^3-10^4$ times smaller than the measured deuteron form factor, implying large soft contributions to the form factor, in agreement with \cite{Isgur:1988iw,Radyushkin:1990te}, suggesting that pQCD should not be used as an explanation for the form factor. The calculation is extremely complicated and a confirmation, or refutation, is desirable."~-R. Gilman and F. Gross~\cite{Gilman:2001yh}

``From the discussions in section 3.8, it is clearly of interest to extend measurements of $A$ to higher $Q^2$. An $ed$ coincidence experiment is straightforward, but prohibitive timewise with present accelerators. The proposed 12~GeV JLab upgrade allows one to take advantage of the approximate $E^2$ scaling of $\sigma_M$ at constant $Q^2$ and high energy~\cite{Petratos:2002wz}. A large acceptance spectrometer such as MAD would be very helpful. Depending on the details of the upgrade, a one month experiment could provide data to $Q^2$ of 8 GeV$^2$."~-R. Gilman and F. Gross~\cite{Gilman:2001yh}

``Within the context of a more realistic dynamical theory, one can use response function separations and polarization observables to enhance the sensitivity to various model dependent \emph{unobservables}, such as momentum distributions, meson exchange currents and medium modifications. One strong recent interest has been to choose kinematics in which the unobserved nucleon has a large momentum; the plane wave approximation shows that this configuration enhances sensitivity to initial-state short-range correlations (i.e. the wavefunction) and possibly quark effects. A number of these experiments have been carried out at various accelerators, but no experiments at JLab have yet reported the results."~-R. Gilman and F. Gross~\cite{Gilman:2001yh}

``The most precise constraint on these [$I=1$ exchange] currents comes from the $d\rightarrow ^1S_0$ transition, and this part of the trainsition is partly obscured by the poor renergy resolution of the existing high $Q^2$ measurements. A new and improved experiment at JLab with higher resolution would allow the threshold $d\rightarrow ^1S_0$ process to be better extracted, with a better resulting determination of the isovector exchange currents. It is also important to determine whether or not there is a minimum near 1.2 GeV$^2$."~-R. Gilman and F. Gross~\cite{Gilman:2001yh}
